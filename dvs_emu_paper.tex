\documentclass[]{article}
\usepackage[utf8]{inputenc}

%opening
\title{A Real-time Dynamic Vision Sensor Emulator using Off-the-shelf Hardware}
\author{Garibaldi Pineda García}

\begin{document}

\maketitle

\begin{abstract}
Vision is one of our most important senses, a vast amount of information is perceived through our eyes. Neuroscientists have performed several studies using vision as input to their experiments. However, computational neuroscience has typically used Poisson-encoded images as spike-encoded visual sources. Recently neuromorphic Dynamic Vision Sensors (DVS) have surfaced, while they have excellent capabilities, they remain scarce and difficult to use.

We propose a software-based visual input system, it's inspired by the behaviour of a DVS, but uses a digital camera as a sensor. By using readily-available components, we believe, most scientist would have access to spike-based visual input. While the primary goal was to use the system as a real-time input, it is also able to transcode well established images and video databases into spike train representations.

\end{abstract}

\section{Introduction}
DVS\\
Carver Mead, Zurich, Sevilla\\

Cameras are in almost every computing device.\\

Processing power today allows for these devices to perform a few calculations on the image stream to convert into a spike representation.

Other models not in real time or require high performance hardware or FPGAs.


\section{Work}
RGB images are gamma-encoded, to better utilize bits and be compliant with CRT monitors.

DVS pixels output the logarithm of their input current.\\

Asynchronous pixel behaviour is approximated via differencing the current and a reference frame. \\

Rate-based output is possible, though it may overwhelm communication channels.

We propose different time encodings (raw value/binary, threshold multiples/binary, threshold multiples/linear)\\

Extra behaviours: Local inhibition reduces bandwidth, persistence of stimuli over frames. 

Adaptive threshold, slow-charging pixels by reducing the threshold.\\
Less parameters to configure.

\section{Results}

Real-time DVS emulation on consumer hardware (Intel i5, 8GB ram)

\section{Conclusion}

GPU computing version would allow for higher resolutions.

Working on ganglion cell/Difference of Gaussian encoding.

\end{document}
